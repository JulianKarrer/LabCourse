
\chapter{Introduction}

Smoothed Particle Hydrodynamics, or SPH for short, has established itself as a particle-based method of fluid simulation in Computer Graphics for its ability to handle free surface flows with complex boundary conditions and its versatility in the materials that can be simulated using a unified framework, amongst other benefits.

This report documents not only the implementation of an equation-of-state based solver and variations thereof, but attempts to frame the problem of fluid simulation from the more general perspective of numerically solving a partial differential equation by covering the main aspects common to such problems: understanding the underlying governing equations, the initial conditions, boundary conditions and the numerical scheme used to discretize the equations and propagate a solution forwards in time. This is reflected in the structure of the report, which first derives the Navier-Stokes equations that underpin the fluid simulation in \autoref{chp:governing-equations}, stripping the governing equations of their mystery. Smoothed Particle Hydrodynamics is then discussed as a possible discretization of the equations in \autoref{chp:sph-discretization}, yielding three variations of a fluid solver presented in \autoref{chp:solvers}, which build up from a simple solver to an iterative version that can simulate large scenes with low compressions robustly, as demonstrated by the title page and \autoref{fig:title-image}. The boundary conditions and initial conditions are discussed in \autoref{chp:boundary-and-initial}, with a focus on the often neglected aspect of how the geometry of the initial sampling of a field can influence stability and isotropy. These aspects, as well as the differing behaviour of the presented variants of solvers, are then analysed in \autoref{chp:analysis}, yielding some insight into the effects that certain choices of parameters can have on the simulation.\\


This report attempts to repeatedly ask \textit{why} an aspect of fluid simulation with SPH is implemented the way it is, instead of solely focusing on \textit{how} it is implemented. Lagrangian methods are contrasted with Eulerian approaches to explain why an especially convenient form of the governing equations can be obtained using this mesh-free method. By deriving the Navier-Stokes equations, the simplifying assumptions they rely on are revealed, both building an understanding as to why the equations consist of their respective terms and how they might be adapted when such assumptions are broken. The question of 'why' in regard to the discretization of fields using SPH is answered by considering what it means to sample a continuum at discrete locations and how a Gaussian-like kernel function with spherical symmetry might be a particularly natural and in some regards optimal choice for this problem. In building up to an iterative solver step by step, the constituent components can be understood and the question of why the pressure computation is split off and iterated upon can be answered.
When considering boundary conditions, simplifying assumptions that allow for single-layer, non-uniform samplings are motivated by again building up from simpler, less powerful methods of boundary representation and considering their respective flaws. It is then asked why field are initially sampled in one lattice or another, why samplings should be regular or pseudorandom and why uniform masses or uniform densities might be preferred when initializing a system, when both cannot easily be obtained in conjunction. Lastly, the analysis attempts to shine a light on why certain parameters affect the fluid simulation as they do, using techniques that analyse the structure in space, behaviour in the frequency domain and shape in parameter-space that each of the solvers and initial conditions result in.