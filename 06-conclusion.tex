
\chapter{Conclusion and Future Work}

In this report, the Navier-Stokes equations have been derived, discretized, initial conditions and boundary conditions discussed and three variations of a solver for the system of partial differential equations have been presented, each building up from a simple solver using an equation of state, through the concept of operator splitting towards an iterative solver with superior stability and incompressibility guarantees. Parameters that affect the stability, numerical viscosity and the quality of the initial sampling of the continuous fields into discrete particles have been analysed.

The generality of the SPH scheme as a method of discretizing and interpolating fields was highlighted, decoupling it from preconceptions about the more specific implementations that commonly use SPH to simulate fluids, such as the density invariant source terms and even Lagrangian frameworks not being related to SPH by necessity but rather through choice. Gaussian-like kernels in particular, which are often taken for granted in SPH literature, were appreciated and possible reasons for their widespread use were given. An uncommon form of an iterative solver that more directly shows the connection between equation-of-state based solvers and more elaborate iterative solvers commonly in use was given, proving that the hurdle towards implementing a solver that can uphold incompressibility might not be as high as it seems.

A particular emphasis was placed on discussing how a field can be discretized to yield initial conditions that prevent aliasing artefacts and how the lattice that a field is sampled with, whether particles of uniform masses or densities are sampled and if a regular or pseudorandomly jittered configuration is chosen can influence the simulation quality in conjunction. This is a section of the report in that could use a more thorough investigation into what causes numerical viscosity and what parameters could positively affect this behaviour.

The frequency of oscillations observed when using the absolute density to enforce the continuity equation was analysed and some relatively speculative claims where made that also warrant further investigation, such as the exact nature of the relation between stiffness, speed of sound, oscillation frequency and the number of iterations of a solver. This might be an area of particular interest, since unwelcome oscillations can be seen as a major flaw in many SPH-based fluid solvers.

All in all, there are many angles from which the observations laid out in this report could be more thoroughly theoretically underpinned and empirically validated or taken as motivation to investigate a perhaps less vigorously researched branch of improvements to fluid simulation with Smoothed particle hydrodynamics, and the possibilities for future work in this field seem endless.
